\documentclass[a4paper, 12pt]{extarticle}
\usepackage{GS7}
\sloppy

\begin{document}
	\def \nocredits {}

\section{Записки с лекций курса MTK{\_}FPGA{\_}2016}


\subsection{17 ноя в 20:35}

% Дмитрий Ходырев

Внимание! В обсуждениях появились две важные темы, касающиеся плана курса, а так же правил выполнения дз.

Спасибо всем, кто посетил сегодняшнюю лекцию. Для тех кто отсутствовал: центральной темой сегодняшней лекции (\#3) были конечные автоматы (FSM). Почитать о них можно в учебнике Д. и С. Харрис (гл. 3.4) а так же в учебнике pong p. chu (гл 5). Покрытые топики верилога: общий подход к описанию FSM, конструкции case, module instances, параметры модуля. Так же рассмотрели способ создания простого тестбенча (clk generation, time delay, \$write(), ...).

В комметах результат нашей работы в классе (разработка конечного автомата для примитивного лифта). Слайдов для этой лекции не было.

Я доделал тестбенч к нашему модулю, который мы написали в классе. Тестбеч получился сложнее чем я планировал. Не обязательно в дз так упарываться, делать функции, кучу текста в логах. Например, для светофора можно просто посмотреть времянки.

Попробуйте запустить сами, не забудьте читать комменты в тестбенче.

ATTACH\\GROUP\\elevator\_fsm.tar.gz

Кстати, я нашел одну ошибку (спорную), которую мы допустили при разработке автомата. После того момента как на вход модуля пришел сигнал "мы приехали", двигатели лифта еще будут работать на протяжении такта. Я счел это неправильным и добавил зануление управляющих моторами сигналов когда сигнал "мы приехали" в "1".

Таким образом, входные воздействия FSM теперь напрямую влияют на выходные. Автомат Мура превратился в автомат Мили.



\subsection{18 ноя в 15:53}

% Дмитрий Ходырев

А вот и анонсированное на 3 лекции домашнее задание.

Это задание долговременное. Начинать делать и сдавать можно уже сейчас, но сделать все к следующей лекции не просим. Но мы надеемся, что к февралю все уже сделают эти задания.

По мере накопления присланных нам рабочих модулей мы будем проводить тестирование на реальном оборудовании в классе (например, если больше половины участников курса пришлют нам работающие светофоры, мы их проверим).

Не забывайте читать правила выполнения дз перед тем как делать (они есть в обсуждениях). Если будет слишком трудно, упрощайте, делайте сначала без параметризации, тонкостей, хитростей.

ATTACH\\GROUP\\FPGA\_LAB\_2.pdf\_old

22 ноя в 23:27
Илья Шарин

Опечатка в задании A3 - десериализатор:
Есть:

data\_val\_i input - Сигнал, подтверждающий, что data\_i и
data\_mod\_i валидны, держится один такт.

Скорее всего, сигнал должен держаться все время, пока по data\_i поступают сериализованные данные

Дмитрий Ходырев

Илья, да, упоминание о data\_mod\_i там явно лишнее. Похоже на ошибку копипасты. Надо переформулировать: Каждый валидный бит посылки на data\_i сопровождается активным состоянием data\_val\_i. Исправил.

Кстати, по умолчанию "активное состояние" это "1", если явно не указано обратное. Например, постфиксом \_n (data\_val\_n), но это редкость.

Нам пора заводить багтрекер для наших дз

\subsection{29 ноя в 22:01}

Дмитрий Ходырев

А вот и два простеньких проекта, которые мы рассматривали на лекции 4.

Попробуйте собрать их в квартусе и посмотреть интерпритацию в RTL Viewer (Tools->Netlist Viewers). Попробуйте так же просимулировать их. У кого пока-что плохо выходит с тестбенчами — можно поиграть с тестбенчами для этих проектов. Добавить какие-нибудь новые сочетания воздействий. Например, что будет если на входе кодера если сразу несколько линий будет активно?

В кодере/декодере нет ничего примечательного, это просто пример применения конструкции for в синтезируемом коде. Её легко понять, мысленно развернув в последовательность объявлений if ... else if ... else if... Если представите, сможете даже без симуляции сказать, что будет если на входе кодера активно сразу несколько линий при таком устройстве.

Реверсивный счетчик интересен тем, что там представлены две версии модуля, с моноблочным описанием и с описанием в два блока. Во втором случае обратная связь регистра счетчика описана в отдельном блоке комбинационной логики. При этом реализуется одна и та же логика (соберите и убедитесь в RTL Viewer). Я прикрепил два скриншота того, что нарисовал мне RTL Viewer (первый — counter, второй — counter v2). Они внешне немного отличаются, но функционально одно и то же.

ATTACH\\GROUP\\coder-decoder.tar.gz
ATTACH\\GROUP\\counter.tar.gz

\imgn{IMG\\GROUP\\Lec4\\counter}{0.7}
\imgn{IMG\\GROUP\\Lec4\\counter\_v2}{0.7}

Дмитрий Ходырев 29 ноя в 22:10

Интересно, что он нарисовал логику синхронного сброса в виде отдельных от самого регистра мультиплексоров. В реальности (Cyclone V) для этого будут использованы внутренние ресурсы логической ячейки (вообще не обязательно). Может быть, это потому, что на этапе сборки нетлиста синтезатор еще не знает, что у ячеек будет встроенный синхросброс. А может быть, у RTL Viewer нет моделей тригеров/регистров с синхросбросом. Об этом в частности как раз собирался рассказывать на следующей лекции.

UPD: я сделал маленький проектик с одним триггром и пытался уговорить квартус использовать для синхронного сброса внутренний ресурс. Но он упорно делал логику синхронного сброса на LUT (см. картинки). Бодался упорно, но пришлось смириться, что из HDL этими нюансами управлять невозможно.

\imgn{IMG\\GROUP\\Lec4\\int_reset_code}
\imgn{IMG\\GROUP\\Lec4\\int_reset_ex}

Ответ нашелся довольно быстро. В первой ссылке по запросу 'altera how to force sclr' в гугле: https://vk.com/away.php?to=https%3A%2F%2Fwww.altera.com%2Fsupport%2Fsupport-resources%2Fknowledge-base%2Fsolutions%2Frd04122012_841.html

Оказывается, квартус не будет разводить синхронный сброс через SCLR в ячейке если не наберется достаточное число тригеров с асинхронным сбросом от этого сигнала. А все потому что он не сможет упаковать триггеры в одну колонку (LAB), если сигналы управления синхронной логикой (CLK, ACLR, SCLR, ENA, ...) будут подключены к различным источникам.

Есть возможность форсировать использование встроенных сигналов управления синхронной логикой (см картинки). Но я бы не стал вмешиваться, квартус лучше знает как экономить ресурсы чипа.
\imgn{IMG\\GROUP\\Lec4\\forced_variant}
\imgn{IMG\\GROUP\\Lec4\\forced_variant_timing}

С ENA (clock enable) ситуация, видимо, такая же. Для индивидуального триггера он попытается не использовать ENA а врубить обратную связь с Q на один из входов LUT (посмотрите на схеме). Если сигнал enable не активен, в LUT Q триггера будет мультиплексироваться на D триггера. Функционально получится то же самое.
2 дек в 16:13

\subsection{2 дек в 16:47}
Дмитрий Ходырев

Добавил слайды по пятой лекции, а так же перевыложил (опять) текст задания #2, так как там снова нашлись ошибки (задание A6). Теперь оставили там только подсчет единиц, нули считать не надо.

Посмотрите комментарии к предыдущему посту, я накопал немного интересной информации по разводке стандартных сигналов управления синхронной логикой. Очень интересно!

Некоторые комментарии по лекции:
Когда я показывал линию задержки я немного упоролся, описав её через цикл for. Все можно сделать намного проще, в одну строчку через конкатенацию. См. прикрепленный файлик delay_line.sv. Если раскомментите второй способ и соберете, то получите то же самое.

ATTACH\\GROUP\\Slides_lecture_5.pdf

ATTACH\\GROUP\\FPGA_LAB_2.pdf

ATTACH\\GROUP\\delay_line.sv



habr.ru/p/281525/



\subsection{7 дек в 17:06}

Дмитрий Ходырев

Немного комментариев по самостоятельным работам.

Прежде всего, хочется прокомментировать популярный А5. Вы обратили внимание, почему в интерфейсах остальных модулей предусмотрен флаг валидности входных/выходных данных, а у этого модуля его нет? По задумке, это намек на то что модуль можно сделать чисто комбинационным.

Предположим, на вход этого модуля подаются данные из другого модуля, а там они защелкнуты на выходе. Это значит, что если срабатывание комбинационки priority_encoder укладывается в период нашего клока, то в конце любого такта будут валидные данные и доп. флаги не нужны (да, и clk_i можно не использовать).

Однако, good practices советуют защелкивать выходные сигналы модулей. И они правы, но для вот таких маленьких модулей с чисто комбинационной функцией можно сделать исключение.

Если вы сделали в А5 регистр на выходе это не ошибка, но этот факт обязательно должен быть описан в доке (readme), то что соотв. валидные данные будут только на следующем такте.

А вот защелкивать входные воздействия в А5 не стоит.

Вообще защелкивать входные данные прежде чем что-то с ними делать имеет смысл, например, если они потом долго используются (как в сериализаторе).

Пока все. Проверяйте иногда комменты к постам на стенке, туда может быть интересная инфа.

\subsection{13 дек в 17:04}

Дмитрий Ходырев

Выложил слайды прошедшей лекции.

Были запросы вкратце описать что было на лекции и что читать тем, кто пропустил. Ответил в топике Синтез.

Следующая лекция (15 числа) последняя в этом году. Основной темой станет баланс speed/area/latancy в проектах. Например, как увеличение latency (конвейер) может радикально уменьшить кол-во занимаемой логики и повысить Fmax.

ATTACH\\GROUP\\Slides_lecture_6.pdf


\subsection{13 дек 2016 в 16:59}

Дмитрий Ходырев

Лекция 6: RAM в Altera. В чипах Altera под которые мы пишем есть ~несколько сотен боков встроенной RAM памяти небольшого объема (десятки КБ). Так же реализовать RAM память можно на логических ячейках общего назначения, что может быть выгодно если эта RAM совсем малого объема. Память бОльшого объема (условно, от КБ) делать на LE не выгодно. По умолчанию синтезатор, анализируя ваш код и видя там описание RAM (может быть разным)
\begin{verbatim}
...
if( write_enable )
mem[write_address] <= input_data;
...
assign output_data <= mem[read_address];
...
\end{verbatim}

попытается использовать встроенную память (http://quartushelp.altera.com/14.1/mergedProjects/hdl/vlog/vlog_pro_ram_inferred.htmы). Однако, если вы в коде просите сделать что-то, что блочная память делать не умеет (например, reset), то синтезатор не сможет использовать встроенную блочную память и разведет все на LE, RAM займет тысячи логических ячеек. Конечно, использование блочной памяти можно "заказать" явно из HDL кода используя альтеровские примитивы (см там же по ссылкам), но мы так не делаем обычно. Еще один вариант это создать готовый RAM модуль в Qsys MegaWizard. По сути это будет тот же примитив (например, altsyncram, но в оболочке с готовой параметризацией).
Далее можно просто набрать в google "altera ram".

FIFO это производная от RAM. https://ru.wikipedia.org/wiki/FIFO
Используется очень часто. Для FIFO так же есть встроенная функция, которая может сгенерировать FIFO-модуль с заданными характеристиками. Пример описания FIFO есть в учебнике Pong P Chu стр. 110.

Советую почитать альтеровские шаблоны в quartus (http://quartushelp.altera.com/14.1/mergedProjects/design/ted/ted_com_insert_template.htm), а так же посоздавать различные вариации RAM и FIFO в MegaWizard, посмотреть что он вам сгенерировал.




\subsection{18 дек в 21:50}
Дмитрий Ходырев

Итоги лекции №7.

Так как на лекцию пришло всего 4 человека, в теорию мы не углублялись и сосредоточились на практике. Те кто не был не волнуйтесь, ничего блокирующего дальнейшее продвижение вы не пропустили, хотя было очень интересно. Слайдов не было.

Темой лекции были основные показатели проектов, с которыми мы имеем дело: потребление логики (area, LE/ALM), максимальная тактовая частота схемы (Fmax, MHz), пропускная способность (throughput что-то/c), задержка вход-выход (latrncy, такты). Чем, когда и как мы жертвуем чтобы вытянуть другие показатели.

Мы взяли за основу кодер, который я показывал на 4 лекции, начали увеличивать разрядность и смотреть, как потребление логики растет в геометрической прогрессии а Fmax падает. Остановились на значении 256->8. Затем мы разработали схему, разбивающую вычисление на два этапа с промежуточным сохранением (2-stage конвейер). Результаты нас порадовали. Мы пожертвовали latency всего на 1 такт и получили большой выигрыш по частоте и ресурсам. Ниже прикреплен допиленный проект, который мы разработали в классе и табличка с результатами.

Осталось только разхардкодить разрядность pipelined кодера. Может быть даже попробовать найти critical path в TimeQuest, сделать конвейер в 3 стадии и дополнить табличку. Кто так сделает, будет просто капитальный красавчик.
ATTACH\\GROUP\\coder_pipelined.tar.gz


Ваня Шевчук
Дим, а в sdc какая была частота настроена?
20 дек в 0:33

Дмитрий Ходырев

Ваня, собирал без констрейнтов, т.к. интересовало прежде всего соотношение Fmax для различных дизайнов при равных условиях. Ты думаешь, коснстрейнты могут значительно повлиять на это соотношение?

Я попробовал сейчас с констрейнтами, но пока это не дало каких-либо значительных результатов. Мб я не все попробовал, на досуге еще покопаюсь.

\imgn{IMG\\GROUP\\Lec7\\Table}

\subsection{19 дек в 12:47}

Дмитрий Ходырев

Схема pipelined шифратора, который мы сделали в классе (см. предыдущий пост)

\imgn{IMG\\GROUP\\Lec7\\Coder}

\end{document}