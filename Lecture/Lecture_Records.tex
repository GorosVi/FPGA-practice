\documentclass[a4paper, 12pt]{extarticle}
\usepackage{GS7}
\sloppy

\begin{document}
	\def \nocredits {}

\section{Записки с лекций курса MTK{\_}FPGA{\_}2016}
\section{Лекция 3}

	Порты в режимах input/ output/ inout.

	\subsection{Параметризация модулей}

		local param (aka) константа. Недоступна извне. По умолчанию int 32-bit.

\begin{lstlisting}[language=Verilog]
defparam - для задания величик констант, заданных через param
defparam my_mod.MY_PARAM = ...
\end{lstlisting}

		"Решётка"

\begin{lstlisting}[language=Verilog]
module my_module #(
	// parameter MY_param=1
	.MY_PARAM (100)
) inst (
	.signal1(signal)
)
\end{lstlisting}

		More - verilog package, define местами (`define MY{\_}DEF 22).
		`MY{\_}DEF - использование



	\subsection{Конечные автоматы}
		Машина (Миле и Мура) - идеализация моделей конечных автоматов

		Лифт.
		%\img{3\01}

		Список состояний
		\begin{itemize}
			\item Стоит
			\item Едет вверх
			\item Едет вниз
		\end{itemize}
		%\img{3\02}

		Входные воздействия
		\begin{itemize}
			\item Приехал на этаж
			\item Кнопка вверх
			\item Кнопка вниз
			\item Верхний этаж
			\item Нижний этаж
		\end{itemize}
%		\img{3\03}

		Выходные данные
		\begin{itemize}
			\item Ехать вних
			\item Ехать вверх
			\item Находимся в движении
		\end{itemize}
%		\img{3\04}

	\begin{lstlisting}[language=Verilog]
		module elevator (
		input	clk_i,
		input	srst_i,
		...
		)
	\end{lstlisting}


%24.11.16
\section{Операторы SV}
	\% остаток от деления
	** Возведение  в степень
	paramrter a =_width;

	reg [a_width-1: 0] addr;
	reg [7:0] mem [2*a_width-1; 0]

	Если не подключить порт имодуля, он получит Z состояние при симуляции

	Симуляция - узнать наличие незвестных (X) значений
	we_have = ^a - если прилетит где-нибудь X -  даст X
	we_have = ^a === 1'bx';

	Приоритет слжения выше, чем у сдвига (но не возведения в степень)


\section{Variable и Net}
	reg a.

%-7.11.16
Блочная память
	Можно включить регистр на выходе
	Можно выставить начальное значение блока при включнии, но сбросить всю нельзя.




\end{document}