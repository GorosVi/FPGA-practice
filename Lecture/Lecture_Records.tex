\documentclass[a4paper, 12pt]{extarticle}
\usepackage{GS7}
\sloppy

\begin{document}
	\def \nocredits {}

\section{Записки с лекций курса MTK{\_}FPGA{\_}2016}
\section{Лекция 3}

	Порты в режимах input/ output/ inout.

	\subsection{Параметризация модулей}

		local param (aka) константа. Недоступна извне. По умолчанию int 32-bit.

\begin{lstlisting}[language=Verilog]
defparam - для задания величик констант, заданных через param
defparam my_mod.MY_PARAM = ...
\end{lstlisting}

		"Решётка"

\begin{lstlisting}[language=Verilog]
module my_module #(
	// parameter MY_param=1
	.MY_PARAM (100)
) inst (
	.signal1(signal)
)
\end{lstlisting}

		More - verilog package, define местами (`define MY{\_}DEF 22).
		`MY{\_}DEF - использование



	\subsection{Конечные автоматы}
		Машина (Миле и Мура) - идеализация моделей конечных автоматов

		Лифт.
		%\img{3\01}

		Список состояний
		\begin{itemize}
			\item Стоит
			\item Едет вверх
			\item Едет вниз
		\end{itemize}
		%\img{3\02}

		Входные воздействия
		\begin{itemize}
			\item Приехал на этаж
			\item Кнопка вверх
			\item Кнопка вниз
			\item Верхний этаж
			\item Нижний этаж
		\end{itemize}
%		\img{3\03}

		Выходные данные
		\begin{itemize}
			\item Ехать вних
			\item Ехать вверх
			\item Находимся в движении
		\end{itemize}
%		\img{3\04}

	\begin{lstlisting}[language=Verilog]
		module elevator (
		input	clk_i,
		input	srst_i,
		...
		)
	\end{lstlisting}


%24.11.16
\section{Операторы SV}
	\% остаток от деления
	** Возведение  в степень
	paramrter a =_width;

	reg [a_width-1: 0] addr;
	reg [7:0] mem [2*a_width-1; 0]

	Если не подключить порт имодуля, он получит Z состояние при симуляции

	Симуляция - узнать наличие незвестных (X) значений
	we_have = ^a - если прилетит где-нибудь X -  даст X
	we_have = ^a === 1'bx';

	Приоритет слжения выше, чем у сдвига (но не возведения в степень)


\section{Variable и Net}
	reg a.

%-7.11.16
Блочная память
	Можно включить регистр на выходе
	Можно выставить начальное значение блока при включнии, но сбросить всю нельзя.


%09.02.17

Avalon.

streaming interface
\begin{itemize}
	\item Source
	\item data >, valid >, [ ready < ], \|, [ start_of_pcaket > ], [ end_of_pcaket > ], [ empty > (== 0) only when EOP == 1;], [ channel > ]
	\item Sink
\end{itemize}

Memory mapped
\begin{itemize}
	\item Master
	\item address >, read >, write >, writedata >, writerequest >, readdata <, readdatavalid <, [byteena], [burst],
	\item Slave
\end{itemize}

Индекс для смешанных массивов  считается слева направо, сначала инекс неупакоыванного массива, потом упакованного

int [a:a][b:b] c [d:d] [e:e]

a = c[d:d][e:e][a:a][b:b]

Чтобы передать значение в функцию по указателю можно использовать тип ref, помимо обычных input и output.

Task: значение выходной переменной задаётся по выходу из task


Процессы
initial
	begin
		fork
			begin
			end

			begin
			end

		loin
	end


%next - 16/03/2017

Названия процессов name:task_this();


Cbvekzwb, можно запустить с контролем выполнения строк проекта - настройка в параметрах тестбенча в симуляции.

B d gfhfvtnhf[ симуляции утфиду сщву сщимукфвпу


Полезная литература

Questa user's manual
Questa Sim reference manual

IEEE Std 1800-2012
Formal language reference

%Next = 23.03.17

ghbvths подклчения JTAG

Boundary scan
fpga for fun - scan example for TAP engine
BSDL file

С3 - randomize - ds,hjity в беспатной версии моделсим
urandom



fix - channel output not compliant with standard


TimeQuest analysis


.qip
.qsf
.sdc
.qip ip-cores import file


svn :

timequest > create clbock > set targets

derive_pll_clocks - read timequest getting started  -  построение constraints для клока, получаемого с PLL
derive_clock_uncertain....


set_false_path  -  снть ограничение тайминга для сигналов, для которых это некритично
set_multicycle_path - разрешить обработку данных на следующих тактах

кузщке ешьштп

and some project settings:

more settings > more analysis &syntesis settings
	remove duplicate registers - под отключение, если используется попуск через регистры для уменьшения fanouut и повышения быстродействия
	disable register merging across hierarchies =>

fitter settings > more settings
	auto register duplication - обратно предыдущему, дуплицирует триггеры, если fanut превышает порог

	assignments editor - опции для отдельных модулей







\end{document}